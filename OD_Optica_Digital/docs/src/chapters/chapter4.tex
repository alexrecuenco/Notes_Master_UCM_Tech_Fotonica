% !TEX encoding = UTF-8 Unicode
% !TEX TS-program = pdflatexshell

\documentclass[../main/main.tex]{subfiles}

\begin{document}
\chapter{Fabrication methods}

slide 5

Cleanroom.

\begin{itemize}
	\item Laminar air flow
	\item HEPA filter for the room
	\item Positive pressure room (prevent things from leaving)
	\item Entry through a 2 door system
\end{itemize}

Temperature control, particle\ldots In Europe, it is usual to aim for $20\pm 1 C$, humidity of $45\%\pm5\%$

slide 6

\begin{itemize}
	\item Large cost
	\item Maintainance
	\item Difficult daily use for workers
\end{itemize}

Habitual equipment. Note that you protect within it, since vapor from chemical or whatever could affect other parts.

slide 8

Diamond end with rotational system that print the edge.

You record it on metal to make the correct shape.

Less than $0.1 \mu m$ precission.

Rotational symmetry is quite decent with these, since rotating around an axis produces a very good trace

You can't mass-produce them like this, you create a negative master that is later used to create the different ones



slide 9

\section{Laser printing}

This needs to be on an isolated anti-vibration place. It has both passive and active anti-vibration methodology

For antivibration: Different anti-vibration fundation, on a passive vibration footing, with active vibration detectors that compensate

We use high-frequency lasers,  large lenses for focal,

Because we have a small focal point, but we have a ``profunddidad de foco'' that is very low. A very small margin of error. (2\mu m)

To prevent errors we use an autofocus, using a different laser that does not activate the resin to use to detect a margin of error and use as feedback to correct the height ($z$) of the laser

This type of fabrication needs to be really quick, since you need to do millions of perforations. For that reason the control system needs to go at a realy high frequency


\subsection{Resine}

slide 10

Photo-resine. A quemical substance that is sensitive and is placed over a surface. Then this is exposed to the laser, that with a very small power reacts

\begin{itemize}
	\item Blue light 365--405 nm.
	\item In the lab you need to use a yellow/red light to prevent affecting the resine
\end{itemize}

\subsection{Printing method}
slide 11

Positive (creates holes), or negative (add material)

Note that the width of the resine is comparable to the focal depth, and you therefore need to deal a lot with the light as it goes there, what is the interference inside that, etc.

\subsection{Photo-resin deposition}

slide 12

Leave a drop of resine on a spinning disk, spinning rate is important to spread it correctly

slide 13

The spinning can create different shapes, you want to find the right speed for it to prevent wave formation

slide 14 doesnt exist; slide 15

\subsection{Auto focus system}

slide 16

Pictures there of a red laser and and the blue laser. You get a signal method.

\subsection{Exposure}
slide 17

You need to calculate what is the exposure time you need based on laser power, etc.

slide 18

Gaussian beam exposition:

Convert the W in J by  integrating and decide that way what is the right width of these thigns

Width of the line is determined by the equation there with the log.

If we print the line with a gaussian beam we might create errors and ondulations at the edge of the line

Therefore, to create a better edge, it is better to create laser wave that has a sharp edge

Slide 19

Image of a line, started and accelerated.

To prevent the acceleration issues, you want to make it move at a constant speed by accelerating to the speed outside the area.

(TODO: Could we just analize acceleration and jerk to create a system that does not require step that accelerates)


Slide 20

This is an example, removing from the DOE


\section{Interferometry printing}

Slide 21

By using interferomtry we can print multiple much sharper lines at the same time with a period $p = \lambda/ 2 \sin (\theta)$

Slide 22

Creating a net with plane wave front interferomtry using a mirror.

You need to allow this to collimate from far away.

\section{Ablation laser}

More agressive. You ablate the material directly, so you need lasers with a lot more power.

These lasers are created as laser pulses.

Material in nanoseconds gets heated, melts, and explodes.


slide 24

thermical process

It is no longer a mW power laser, it is a 10W power or similar

Q-switching. Pulse frequency around 20 kHz.

This is a very direct technique but it works directly and much easier.

They are good for prototyping, and are flexible in use. They work both for metal and glass.

Slide 25

Lab here,

Describes teh current method. Using a non-linear crystal to split in 3, dissipating energy.

Slide 26

Description of lines over steel and glass.

If we have a 3$\mu$m focal point, but we want to make a 20$\mu m$ line. We can move the object further up, although the energy focus will be much smaller, so we will need to go slower.

In glass we can record inside of it by focusing inside. Look that the dots are created with a certain speed.

Slide 27

Fresnel lense over steel.

In a metal, the topography, even if polished, will have some ondulations.

You would use them for components that need to sustain large stresses and forces

Slide 28--29

Electron beam, "e-beam". Best way to do it to do a good method, really good nm scale resolution

Note how lenses are replaced with magnets and other stuff

Costly, and time expensive

Slide 31

Laser-beam is low cost, e-beam is more expensive, takes more time, but better precision


Slide 32

To have a seasaw, we want only order 1. It has a 100\% efficiency when well done. Having a very tight relationship between $\lambda$, $\theta$ and index $n$.

If you have fabrication errors, you can have other orders appearing, which therefore reduces the efficiency of the order 1.

Note that at a perfect
\begin{itemize}
	\item Film thickness variation
	\item Over stretching  (When a=T, efficiency is 1)
	\item Swell
	\item Imperfection of the shoulders
\end{itemize}


\section{Replication and characterization}

Slides \emph{Tema 4b. Replicado y caracterización}


Replication of ODEs are important to reduce cost through economies of scale

\subsection{Mask replication}
Slide 5

You create a 1st-gen replication, then submasters, 2nd gen, etc.

This reduces the tear and wear on the first master, which is very expensive to produce

\subsection{Replication In plastic}

Slide 6

\begin{itemize}
	\item Good resolution, resolution of $100 nm$
	\item Easy to mold
	\item Cost effective. The mold can be expensive, but replication is quite cheap
\end{itemize}

Plastic with time becomes yellow-ish

\subsubsection{Model inyection}

Slide 7

Policarbonato (PC) and Polimetil Metracrilato (PMMA)


Slide 8--9

Insert in the mold the warm liquid, and then cool it down.

Low optical quality compared to master

Precision tuning is required, temperature of liquid plastic, screw speed and force, etc

\subsubsection{Hot embossing}

Slide 10--11

Press a mold with the plastic at warm/high temperatures

Temperature of plastic must be above cristalization temperature

For holograms in credit cards and similar

Material policarbonatos, polietileno, PVC, PTFE, PMMA

\subsubsection{UV stamping and fuse}

Slide 12--14

Similar to hot embossing but with liquid plastic. Expose later to UV to polimerize

It is durable, especially quimically and transparent

\subsubsection*{Silicon molds}

Slide 14

Gets to nm precision.

Slower due to NM precision

\subsubsection*{Comparison}
slide 16


\subsection{Photolithography transfer}

slide 17

This is good for BINARY  nets. It doesn't wark for other networks.


I get a really good master with an electron beam in "cromo"

Then, I want a submaster.

I press a new lythography against the master. Place them in contact, and then you expose it to light.

After light exposure it becomes soft, then I can remove it.

Then I do a chemical attack, that will attack the glass. (Precision control of timing of the chemical attack). The chemical attack attacks the glass and not the cromo

Then, I attack the cromo with a different chemical attack. Leaving just the glass.


\subsubsection{Multi-level photolitography transfer}
Slide 18

Use two masks. One gives two levels, the other gives the other two levels

It is like describing each level as a binary level. Therefore, the one that handles the first ``digit'' needs to be twice as big. That way we can do all four levels (00, 01, 10, 11)

Since this is a binary method, to make $N$ levels, require $\log(N)$ number of masks

\subsubsection{photolitography transfer complexity}

Slide 19--21

One of the biggest complexity on the physical side is aligning.

Aligning the masks is done with some crosses we print as well on the masks (See slide 20)
And then we use a microscope to align all of them.
4 crosses are usually added to have good alignment.

You can also do it with a diffraction net, which will cancel each other

\subsection{Projection systems}
Slide 22

When touching our master we can get issues with damage and tear to the master.

We therefore can add a gap. If we add a 1mm gap on a 40mm resolution binary mask, that is ok. The aberration of light and spread will not affect it
However if you have really small holes, you will need a lens to align light, to do it well (Even though it will invert it)

This is a projection method

\subsection{Contact and proximity}

slide 26

Effects bad

...


\subsection{Lithography with grey levels}

By adding a small mask with small holes within each hole it will be continuous


\section{Thermal reflux/reflow}

Slide 28

Incident UV light with mask, create structure, then melt it to create microlenses

See Slide 29 for extra information on that.


Slide 30

Similar techniques:

Reactive Ion Etching.

In low pressure, photoresin is usually transferred to the silicon structure

Low pressure and fluoride creates strong edges that creates really good quality

Usually use for resin but it can be used to create masks

Look at the precision on slide 32


\section{Characterization}

Slide 33

Parameters that are miportant to characterize our system

Slide 34

Techniques for characterization.

Physical probing ---

Profilometry Good for finding surface but they can wear the components

Tunnel microscopy

Slide 35

Up to half a micrometer with photonic optics. After that the wavelength of the light gives issues. When the size is comparable to the wavelength. In practice this limit is reached around 2--3 $\mu m$


Slide 36

Ilumination techniques.

You provide light laterally.

And I understood nothing from here to slide 38


Slide 39

\subsection{Force atomic microscropy}

We detect a ven-der-Walls force between a tiny tip (AFM cantilevel and tip)

We have a photodetector, and try to maintain the oscillator signal from the tip constant.

To measure the surface we keep moving the tip up/down trying to maintain the oscillator signal constant.

This information allows us to later reconstruct the image

Piezoelectric systems used for displacement, since mechanical displacement systems are really hard to achieve such small movements. With piezoelectric elements, they can move a few nanometers that are controlled applying an electric current


\subsection{Confocal microscropy}

Slide 41

We have two points that are separate enough to not interact

With the piezoelectric eleemnts, as we move in z we can find the maximum of a point, with nm movements. (We find the maximum in comparison to another one, seeing which is higher)

If the second point is at a different height it will not be at the maximum, we can therefore infer the height difference

For each of the points we can do a $z$ search to find the maximum. (When they are \emph{exactly} on the focal)

This would be a 3D search. It takes a long time to get enough precision

Now, if we do this with one point it takes forever. We can do multiple passes at the same time by having enough points separate enough. Speeding up the process

Slide 42

We can go further in speeding up, we can do lines instead of points \href{https://www.youtube.com/watch?v=QFtZFbug1SA}

So we only do the scan in one direction

It still works since each line is far enough to not interact with each other. We look at each point on the line in reference to another line to do another calculation, allowing us to reduce the 3D scanning problem into a 2D scanning problem

Slide 43--44

\subsection{Electronic microscropy}

Really high precision, doing scans.

\subsection{Mach Zender interferometry}

Slide 45

Place the surface at one place on an interferometry two beam.

We know that the phase has changed when placing this object on one of the beams.

We can see the phase differences and discern from that the height difference. However, we don't know which places are holes and which are not, we only know they are different phases, not if it is because of more or less material

To fix that, we do 4 measurements with a piezoelectric attached mirror

Each of those, with a $\pi/2 $ phase from each other.

We get then measures related to $h_1 \propto \sin \Delta$, $h_2 \propto\sin( \Delta + \pi/2) = \cos \Delta$,  $h_3 \propto\sin( \Delta + \pi) = -\sin \Delta$,  $h_14 \propto\sin( \Delta + 3\pi/2) = -\cos \Delta$


We then obtain

\begin{equation}
	\tan \Delta = \frac{h_2 - h_4}{h_1 - h_3} \implies \Delta = arctan\left(\frac{h_2 - h_4}{h_1 - h_3}\right)
\end{equation}

This procedures recovers the sign, so we can know if the shift was in one direction or another


\subsection{Twymann-Green interferometry}

Similar.

We calibrate the actual




\end{document}
