% !TEX encoding = UTF-8 Unicode
% !TEX TS-program = pdflatexshell

\documentclass[../main/main.tex]{subfiles}

\begin{document}
%\setcounter{chapter}{1}
%\setcounter{section}{1}
\section{Homework}

\emph{14 de Octubre: entrega de problemas 2.8, 2.11 y 2.19}

\section*{Problem 2.8}


Which of the following filters can be used in the Fourier domain to enhance edges on an image $f(r)$ (where $u$ is the frequency vector, $H(\u)$ is the frequency response)

\begin{enumerate}
	\item $H(\u) = (\FT{f(\r)}(\u))^{*}$
	\item $H(\u) = \abs{\FT{f(\r)}(\u)}^{-1}$
	\item $H(\u) = \abs{\FT{f(\r)}(\u)}^{-2}$
	\item $H(\u) = e^{-u^{2}/a^{2}}$
	\item $H(\u) = \frac{1}{1+\flatfrac {u}{|\u_{o}|}}$ where $u = \abs{\u}$
\end{enumerate}

% Pasa alta para detectar edges.




The filter will be applied to the signal $f(r)$ by

\begin{equation*}
\FT{f(\r)}^{-1}\left[F(u)H(u)\right]
\end{equation*}

Let's see how they work for each of our statements


\begin{enumerate}
	\item
	\begin{align*}
	g(x)
	&=  \FT{\left[F(u)H(u)\right]}\\
	&=  \FT{\left[F(u)F^{*}(u)\right]}\\
	&=  \FT{\left[\abs{F(u)}^{2}\right]}\\
	\end{align*}

\end{enumerate}

\section*{Problem 2.11}

To filter
\begin{equation*}
	f(x) = \sin(\pi/2) + \cos(2\pi u_{0}x + \pi/3) + \sin(10\pi u_{0} x)
\end{equation*}
A system is used that acts on $f$ and is described with

\begin{equation*}
	g(x) = a \int_{-\infty}^{\infty} %
	f(\xi) sinc^{2}\left[a(x - \xi)\right]\dd{\xi}
\end{equation*}

\begin{enumerate}
	\item Find the Fourier transform and represent the \emph{espectro de potencia de la señal de entrada}
	\item Which is the \emph{function de transferencia del sistema}
	\item suppose you want to remove the highest frequencies (they are noise). What value does $a$ such that $g(x)$ provides more contrast? Determine the filtered signal in that case

\end{enumerate}

\section*{Problema 2.19}

\emph{Una imagen esta corrompido con el ruido ``sal y pimienta''. Usando el programa ImageJ (véanse la diapositiva final de la clase convolución y Correlación) elegir un filtro (entre los filtros Gaussian Blur, Mean o Medium) más adecuado para la reducción del ruido. Presentar las imágenes (original, con el ruido, filtradas) y comentar su elección}

\end{document}
