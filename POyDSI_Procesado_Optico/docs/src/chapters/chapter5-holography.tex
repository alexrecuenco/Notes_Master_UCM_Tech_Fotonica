% !TEX encoding = UTF-8 Unicode
% !TEX TS-program = pdflatexshell

\documentclass[../main/main.tex]{subfiles}

\begin{document}

\chapter{Holography and numerical calculus of Fourier Transform and diffracion integrals}
Slide 3

\begin{itemize}
	\item Obtain amplitude and phase information
	\item Two steps
	      \begin{enumerate}
		      \item Formation and registration of interference fringes (hologram)
		      \item Reconstruction of object beam from hologram
	      \end{enumerate}

\end{itemize}


\section{Different configurations}

Slide 7

In-line holography. If the reference wave is parallel to the object

Off-line holography. Reference wave is inclined with respect to the object one. Highly coherent ilumination is required.


Slide 8


Fourier domain filtering (With a 4f system, you can filter in the middle the fourier transform of the system)

We filter what we don't want. We can then incline and get the fringes
Slide 9

Gabor holography

Object and reference are on the same plane, you can't do the same as with the previous one


Slide 10

With an opaque F in a white background,

Can I do the same reconstruction if it is a transparent F on an opaque screen?
Answer is no, reference is on the same line shown and reconstruction is harder

Slide 11

How to fix this issue?

You make the reference wave separate to the holographic wave

In that case, it can reconstruct better both images

\section{Phase recovery interpherometry }
Slide 12

Michelson

Superposition of object wave with reference wave [eq 1]

We can recover phase, controlling phase shifting.
We can recover phi measuring 4 images at 4 different phase shifts. (See [eq 2] and [eq 3])

We need to be able to move the object with a lot of precision

\section{Digital Holography advantages}

slide 14

\begin{itemize}
	\item Reconstruction of object phase, quantitative phase images
	\item Object refractive index recovery. RI instead of staining
	\item Reconstructing with reading beams.
	\item
\end{itemize}


slide 15

Digital creation of interference.

Circular beams with same shape and different phases.

These would provide different things to trapped particles.
Smooth phase change would give continuous, while phase change jerkiness would cause acceleration/deceleration

Generate the beam digitally and reproduce it analogously

Slide 16. Experiemnt on trapping using these.


TFM. Digital holigraphy, thinguies and other things. Could be a Master thesis if talked about.


\end{document}
