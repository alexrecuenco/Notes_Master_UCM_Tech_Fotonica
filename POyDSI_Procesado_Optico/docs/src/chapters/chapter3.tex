% !TEX encoding = UTF-8 Unicode
% !TEX TS-program = pdflatexshell

\documentclass[../main/main.tex]{subfiles}

\begin{document}

\chapter{Image formation}

Slides 1--36 missed class

\section{PSF of a coherent system}

slide 35

slide 36

...


\section{Coherence influence}

slide 41

Exit signal depends on the correlation function $\Gamma = <f f^*>$

slide 42..43


\subsection{Optical transfer function}
slide 44

Describes the system with incoherent ilumination

Autocorrelation

slide 45, example

\section{Aberration}
slide 46

We have not considered yet aberration on systems

\begin{itemize}
	\item Chromatic
	\item Spherical

\end{itemize}

slide 47

Out of axis aberration. Off

Slide 48

Field curvature/distortion

Slide 49

Aberrated out of axis are not LSI in the focal plane

50

Aberration don't increase MTF

51

Something to look, I guess

52

Coherente/incoherente systems

\begin{itemize}
	\item A monocromatic point source is coherent
	\item An extense quasi-monocromatic source
	      \begin{itemize}
		      \item Coherent with $\theta_{source} < < \theta_{lens}$
		      \item Incoherent with $\theta_{source} > > \theta_0 + \theta_{lens}$

		      \item Partially coherent in other cases ...
	      \end{itemize}
\end{itemize}

Slide 53

Correlation transform

Slide 54

Van Cittert-Zernike theorem

Espacially incoherente source, $\Gamma(\r_1, \r_2, 0) = q \delta(\r_1-\r_2) I((\flatfrac{\r_1+\r_2}{2}))$. Where this is fressnell approx

Then over the transmission plane, $z$, then... (eq)

Van Citter-Zernike theorem. The ocrrelation function $\Gamma$ at a distance $z$ of an inchonerent source is the Fourier Transform of the intensity distribution of the source.

The espacial coherence increases with propagation over a homogeneous  media


Slide 55

The length of coherence at a distance is approximately same order of magnitude as $l_c = \lambda z_l \flatfrac{z_1}{s}$

We can also use filters to remove decoherance in space or color


Slide 56

Resolution criteria

\begin{itemize}
	\item $besinc^2(r) $ for a circular PSF
	\item Minimum sepration for resolution $l = 0.61 \frac{\lambda z_i}{a}$
	\item Non-paraxial
\end{itemize}

Note how these are estimations, our actual resolution will be worse than this perfect situation


slide 57

Example

An incoherent source might have rayleight length, blah blah

However, when they are coherent, if they have the same phase, we can't resolve them, when the phases are opposite we have really good resolution
In other circumstances we might not have resolution


slide 58

An slightly unfocused edge

See the difference between coherent and decoherant incident

slide 59

Coherent source

See the image comparing both, and showing the noise

\begin{itemize}
	\item Advantages
	      \begin{enumerate}
		      \item Visalize transparent object with contrast method
		      \item Cuantitavive phase
		      \item Holographic imaging
		      \item Speckle image forming
	      \end{enumerate}
	\item Disadvantages
	      \begin{enumerate}
		      \item Speckle (interference noice)
		      \item Radiation sources
	      \end{enumerate}
\end{itemize}


\section{Speckle}

slide 60

When a difused object is shined with radiation that is especially and temporally coherent, it contains large contrast areas (speckle) that do not relate to the observed object

Speckle is a diffraction pattern of a coherent beam reflected on a surface.

...

slide 61

who knows really

Just a bunch of randomly placed dots, each of them create a phase difference, all of them create an ensemble that affects the source

slide 62

In accordance to the central limit theorem, the interfearence of different points

The distribution will be a Rayleigh distribution, with $\sigma_I = I^{<average>}$


slide 63

Dynamic speckle, when something moves

$K = \sigma_l / <I>$ measures contrast

K is reduced with speed and with camera time exposition


slide 64

Use of speckle?

non-line-of-sight imaging

Speckle is very important for this

slide 65

Conclusions

The same system has multiple intensity

Characterize the system before using it.

Each year new imaging techniquest
\end{document}
