% !TEX encoding = UTF-8 Unicode
% !TEX TS-program = pdflatexshell

\documentclass[../main/main.tex]{subfiles}

\begin{document}
\section{Fourier analysis}

\subsection{Bases of vector spaces}

Given a dot product, and a complete set of functions ${v_{n}}$ --- for a certain definition of complete --- we can describe any function within that.

Assuming our base is orthogonal under a certain dot product, $\langle v_{i},v_{j}\!\rangle = \delta_{ij}$, When that is possible, each vector will be able to be described,

\begin{equation}
f(x) = \sum_{n}a_{n} v_{n}
\end{equation}

And you can find each parameter using the linearity of the dot product,  because $\langle f,\! v_{n}\rangle = \langle v_{n},\! v_{n}\rangle a_{n}$, therefore

\begin{equation}
a_{n} = \frac{\langle f,\! v_{n}\rangle}{\langle v_{n},\! v_{n} = \frac{\langle f,\! v_{n}\rangle}\rangle}
\end{equation}




\subsection{Fourier}


\subsubsection{Fourier series}

Fouriere series are the separation of periodic functions based on the complete sets  of periodic functions with period $2\pi$, together with the \emph{complete set} $\left\{\sin (nx), \cos(nx), 1 | \,n \in \N\right\}$.

	In this case, if we look at the range  $[-\pi,\, \pi]$, we get with our complete set. We can first normalize the vectors finding the square norm of each of the base items

	\begin{align}
	\fourierint{\sin^{2}(nx)} &=\fourierint{\frac{1-\cos(2nx)}{2}}  &=\pi  \\
	\fourierint{\cos^{2}(nx)} &=\fourierint{\frac{1+\sin(2nx)}{2}}  &=\pi  \\
	\fourierint{1} &&=2\pi
	\end{align}


	Remember that in this base $\langle v_{i},v_{j}\!\rangle \propto \delta_{ij}$, they are not unitary vectors

	Then, if we associate the following parameters

	\begin{align*}
	a_{0} \iff {1}\\
	a_{n} \iff {\cos nx},\, n \in \N\\
	b_{n} \iff {\sin nx} ,\,n \in \N\\
	\end{align*}

	Any period function with period $2\pi$ that is normed under our basis can therefore be expressed as

	\begin{equation}
	f(x) = a_{0} + \sum_{n}\left(a_{n} \cos (nx) + b_{n} \sin(nx)\right)
	\end{equation}


	Where
	\begin{align}
	a_{0} &= \frac{1}{2\pi} \fourierint{f(x)} \\
	a_{n} &= \frac{1}{\pi} \fourierint{f(x) \cos(nx)} \\
	b_{n} &= \frac{1}{\pi} \fourierint{f(x) \sin(nx)}
	\end{align}

	Then, we can do a change of variable to change from period $2\pi$ to any other period $T$. If we do the transformation $\frac{x}{2\pi} = \frac{y}{T}$.

	With the transformation, we can finally describe


	\begin{align*}
	a_{0} \iff {1}\\
	a_{n} \iff {\cos 2\pi \nu n y},\, n \in \N\\
	b_{n} \iff {\sin 2\pi \nu n y} ,\,n \in \N\\
	\end{align*}

	And renaming $y$ to $x$ again --- because we won't be using the $x$ variable anymore --- any period function with period $T$ that is normed under our basis can therefore be expressed as

	\begin{equation}
	f(x) = a_{0} + \sum_{n}\left(a_{n} \cos ( 2\pi \nu n x) + b_{n} \sin( 2\pi \nu n x)\right)
	\end{equation}


	Where,
	\begin{align}
	a_{0} &= \frac{1}{T} \fourier[-T/2][T/2]{f(x)} \\
	a_{n} &= \frac{2}{T} \fourier[-T/2][T/2]{f(x) \cos(2\pi \nu n y)} \\
	b_{n} &= \frac{2}{T} \fourier[-T/2][T/2]{f(x) \sin(2\pi \nu n y)}
	\end{align}


\subsubsection{Fourier transformation}


Continuous form of the fourier series, in which the bases is the vectors $e^{i\nu x}$ and the integration is over $\R$, under a distribution definition of integration that allows improper integrals (It can be understood more precisely with lebresgue integration if needed)

\subsection{Convolution}

\subsubsection{Properties}

\subsection{Signal sampling}

We can reconstruct a  signal based on sampling

\begin{align}
f(x) \implies& F(u)\\
f_{sample} \implies& F_{sample}
\end{align}

\subsection{Fresnel transformation}

\begin{itemize}
	\item

\end{itemize}

\subsection{Hilbert transform}

\begin{itemize}
	\item \begin{equation}
	TH[f(x)]](u) = H(u) = \frac{1}{\pi}\int_{-\infty}^{\infty}\frac{f(x)}{u-x} dx = f(u)\conv \frac{1}{\pi u}
	\end{equation}
	\item $H(x) = f(x) \conv \frac{1}{\pi u}$
	\item Fourier transform, something something something


\end{itemize}

\subsection{Analytical signal}

\begin{itemize}
	\item Negatvie frequencies are redundant. The fourier transform of a real signal is hermitic. (Real part is  even and imaginary part is odd) $F(-u)= F^{*}(u)$
	\item Analytical signal
	\begin{equation}
	a(x) = f(x) + i \, TH[f](x)
	\end{equation}
	\item Fourier transform of an analytical signal only has twice the positive frequencies

\end{itemize}

\section{Missed 20 minutes}

\emph{Densidad espectral de potencia}

\section{Wiener-Khinchin theorem}

slide 56 ()

If it is time independent $F^{*}(v_{1})F(v_{2}) = A(v_{2} \delta(v_{1}, v_{2}))$

slide 57

white noise, pink noise, brown noise (brownian)


slide 58

In signal processing Wiener-Khinchi theorem.

Michelson detector.  Signal received $\implies$ Spectrum (through FFT). Find movement.


\subsection{Summary}

\begin{enumerate}
	\item Convulucion operacion filtrado invariante espacialmente.

\end{enumerate}






\section{Linear system characterization}

Slides ``Caracterizacion de sistemas etc''



slide 3

Linear system have signal that, given arbitrary blu blu

slide 5 non-linear systems

\begin{itemize}
	\item $P=V^{2}/R$
	\item Fidelidad sinusoidal (peak detection and other thinguies )

\end{itemize}


slide 6 (2,4,5 are linear)

slide 7 (casual system.)

If response depends only on $f(t)$ for values of $t<\leq t_{0}$

(1,3) casual system

(2,4) not (2 is an average, 4 is time reversed)


\subsection{Characterize}

Expand signal, (Hermite, Laguerre, Zernike, etc) polynomials.

\begin{equation}
f(x,y) = \sum_{n=1}\sum_{m=1} a_{nm} \phi_{nm}(x,y)
\end{equation}

Since it is linear, characterization is linear, and we can base response based on the sum

\begin{equation}
g(x,y) = S\left\{f(x,y)\right\} = \sum_{n=1}\sum_{m=1} a_{nm} S\left\{f\phi_{nm}(x,y)\right\}
\end{equation}

\subsubsection{Impulse response}

% TODO put it in main,
\newcommand{\hh}{h(x,y,\eta,\phi)}

slide 9

Due to linearity we can divide every function in terms of each individual impulse..

PSF is Point Spread Function


\begin{equation}
h(x,y,\eta,\phi) =  S\left\{\delta(x-\eta,y-\phi)\right\}
\end{equation}

Response of system in $(x,y)$ when an impulso occured in $(\eta,\phi)$

\subsubsection{Image flood}

slide 10

\begin{equation}
g_{\textrm{flood}}(x,y) = \iint h(x-\eta,y-\phi) \dd{\eta}\dd{\phi}
\end{equation}

if $g_{flood}$ depends on $x,y$ it is non linear posterior to g

Sensitivity $S_{pi}$

\begin{equation}
S_{pi} = \iint h(x-\eta,y-\phi) \dd{x}\dd{y}
\end{equation}

slide 11,12 graphic example


slide 13

Linear shift-invariant systems, (LSI system)

if $\hh = h(x-\eta, y-\phi)$

Output signal is the convolution of h with the input signal. Not all linear systems are spacially invariant

slide 14

description throgh a matrix

\begin{equation}
g = M\times f
\end{equation}

Toeplitz matric,



slide 15

Problem 2.10

TODO: Move to problems

NOTE: To calculate the PSF we change f for delta and calculate
\begin{enumerate}
	\item it is linear, since the question is linear with respect to f.

	Is it invariant with respect to shift? No, it can't be expressed in a convolution form

	\item Non-linear, it does not have PSF because that is only for linear systems

	\item It is linear, what is the PSF? the exponential

	It is LSI

	\item Fourier transform,

	PSF is the exponent squared,

	non lsi

\end{enumerate}

slide 17, example

LSI have constant flood


slide 18

LSI system con aumento 1



$f(r) = m^{-2}f_{e}(r/m) \implies h(r, r') = m^{-2}f_{e}((r-r')/m) $


PSF $g = \int \dd{r'} f(r') h(r-r')$


slide 19 funcion de transferencia de LSI


\begin{itemize}
	\item Systems with in/out relationsihp that relte through a convolution are linear and LSI
	\item Their ext
	\item In fourier domain it is simply a product
	\item The transfer function $H$ of the system is the fourier transform of the PSF

	$H(p) = TF\{h(r)\}(p);\ p=(u,v)$

\end{itemize}


slide 20

If PSF is real, then $H(p) = H^{*}(-p)$ can be expressed in terms of positive frequencies. IF PSF is even and real, PSF is also real and even


NEXT DAY

slide 22

Normalized transfer function $OTF = \frac{H(p)}{H(0)} = MTF(p) e^{i\,\textrm{PTF(o)}}$


slide 23

contraste, Michelson

\begin{equation}
M_{f} = \frac{f_{max} - f_{min}}{f_{max} + f_{min}}
\end{equation}

Out signal of LSI

\begin{equation}
g(x_{0})
\end{equation}
Note $a$ will be transfer at point $0$, $b$ will be transfer point at $u$,

slide 24

Entry signal


Exit signal

If $\phi\neq0$ then there will be a de-phasing, not just an attenuation . Remember that a phase shift is a shift on their harmonics.

slide 25

When MTF characterizes contrast, when there is a phase thingy then it becomes more grey and harder to see


slide 26

Linear spread function. Based on how the system reacts to a line of delta density

If the excitement axis is $y$ $f(x, y) = \delta(x)$. Its LSF($l$) is related to its PSF($h$)



slide 27 (you go so fast I have to say)

Response to a perfect edge

Step function $s(x)$

The LSF is the derivative of the convolution of the spread function and stimulus (the response to the edge).

slide 28

slide 29

Measures of MTF

Direct (patterns inusoidal, barrs, chirp cos)

Indirect measures

Por PSF, LSF or SR

slide 30

drawing of stimulus versus luminance of image. and a bunch of explanations and some expansino in term of harmonic (only odd ones, even cancel for the step function)


slide 31

I dont know

slide 32

PSF provides more information from the system than the img resolution.

The limit can be calculated empirically. Half width, Rayleigh criteria, etc

slide 33

Resolution, width at half the PSF

Simple moethod but faluable

slide 34

Rayleigh

slide 35

Sparrow

Note how the calculation on the slide relies on both being same hegihts, the real resolution of the system will be lower than this limit, since in general both delta sources will have different heights

Note how all these are just criteria, it doesnt change our image


slide 36

Superresolution

Resolution better than on the classical limit.

\begin{itemize}
	\item Extrapolation based on sampling theorem
	\item Iterative extrapolation (Gerchberg)
	\item
	\item Engineering of PSG

\end{itemize}•


slide 37

If a signal is compact (bound), we can describe it by a sampling based on the interpolation theorem. We can therefore describe F based on samples (see slide for eq)

Presumably, once we know H, the signal spectrum something something.

We can only know frequencies on a range


slide 38

Noise

Additive noise (gaussian, thermic)

Signal dependent noise (Poisson/shot noise, )

slide 39

Contrast to noise ratio, CNR

slide 40

Deconvolution, debluring

Deconvolution is a computational technique to partially compensate with the distortion.

Picture is ncie on the slide

``Inverse convolution''

slide 41

Inverse filter (naive filter)

Use deblurring with  h a 2-pixel gaussian

slide 42

In fuirier doman

\begin{align}
G = FH + N\\

f(r)? = TF^{-1}\left(\frac{G(u)- N(u)}{H(r)}\right)
\end{align}

Non unique solution

Unstable

slide 43

something fun, deregularized

\end{document}
