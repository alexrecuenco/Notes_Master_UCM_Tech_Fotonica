% !TEX encoding = UTF-8 Unicode
% !TEX TS-program = pdflatexshell

\documentclass[../main/main.tex]{subfiles}

\begin{document}

\chapter{Radon transformation}

Slides ``\emph{Transformadata de Radon}''
\section{Introduction}
slide 3

Beer's law $I(l) = I_0 e^{- \int_0^l \eta(x) dx} \implies \ln(I_0/I) =\int_{-\infty}^{\infty} \eta(x) dx $

Projection taken over parallel lines would mean it is a parallel projection

Projection taken from the projection of one point, fan-beam projection and cone-beam projection (3D)

\subsection{Tomography in medicine}
Slides 4--6

Nobel prizes, whatever

Brain tomography using line scans in multiple anlges

\section{Definition of Radon transform}

Slide 7, see picture

Basically, we define a line from detector to source, and the angle of this line changes as well as we go through every line

We then integrate the line $\vec(r)\vec(n) = s$ where $\vec{n}$ is the normal of the line of measurement

Under that, the transformation is
\begin{equation}
	R(s, \theta) = \iint f(\vec r) \delta(s - \vec{r} \vec n )\dd{\vec r}
\end{equation}

With the parametric definition of the line,

\begin{equation}
	R(s, \theta) = \int \dd z f(s \cos \theta - z \sin \theta, s \sin \theta + z \cos \theta )
\end{equation}

The full set of R are the Radon transformation

TODO Matrix slide 8, do it

\subsection{Senogram}
slide 9--10

The projection has coordinates $s, \theta$ in an interval $\theta \in [\theta_{0}, \theta_{0} + \pi]$

The transform of a point, $f = \delta(\vec{r}-\vec{r_{0}} )$ will be a line in $(s, \theta)$ space of poles. When the point coincides with the measure line for each angle

\begin{equation}
R(s, \theta) = \delta(s - r_{0} \cos (\theta - \theta_{0}))
\end{equation}

Where $\theta_{0}$ is the line original

Therefore, $s = r_{0} \cos(\theta-\theta_{0})$ will define the absorvation line.


Slide 11

When you have multiple points, each of these lines overlap


\section{Radon transform properties}

\begin{itemize}
	\item Linear
	\item Periodic, period $2\pi$
	\item Symmetry, $R(s, \theta) = R(-s, \theta + \pi)$

\end{itemize}

Flow conservation

\begin{equation}
\iint f(x,y) \dd x \dd y = \int R(s, \theta) \dd s
\end{equation}

Rotation. Given polar coordinates $f(r, \phi)$ some transformation

Scaling theorem

Displacement

Homeomorphic through convolution

Space limited signal

\section{Shape reconstruction}
\subsection{Fourier transformation}

slide 14

Central slice theorem: Can you recover the initial signal given a radon transform?

Fourier transform with respect to s is equivalent to the bidimensional transform of $f$, where the vector $\nu \vec{n}$ is there instead of the spacial frequency
\begin{equation}
TF_{s} =///
\end{equation}

slide 15

F()

\begin{equation}
F(\nu \cos \theta, \nu \sin \theta) = F(u, v) = TF_{s}{R} = \int_{\R} \dd{s} R(s, \theta ) e^{-i2\pi \nu s}
\end{equation}

\subsection{Procedure to reproduce}

slide 16

\begin{itemize}
	\item How many projections need to be taken for reconstruction
	\item Assume a signal is compact. $f = 0$ for a certain $x^{2}+y^{2} \geq T_{0}$ and its fourier transform is space limited as well
	\item Due to sampling theorem
\begin{align}
\Delta x \leq 1/2F_{0} & \Delta y \leq 1/2F_{0} \\
\Delta u \leq 1/2T_{0} & \Delta \nu \leq 1/2T_{0} \\
\end{align}
Where $T_{0}$ is the size of the object, and  $F_{0}$ is the maximum frequency bound

\end{itemize}

slide 17

Fourier method of reconstraction

Interpolation, measure a net in the polar thingy. low resolution for high frequencies


slide 18

The transform can also be expressed instead of ranges $0, \infty$ and $0, 2\pi$ due to Radon's symmetry, $0, \pi$, and then we do it twice. so also $-\infty, \infty $

Slide 19

\begin{equation}
f(r) = \int_{\R} \abs{v} \dd v \int_{0}^{\pi}e^{i2\pi v(x \cos \theta + y \sin \theta)} R^{hat}(v, \theta) \dd \theta
\end{equation}

This form is basically the frequency filter, using the filter version using the frequency filter with a response $H(\nu) = \abs{\nu}$


\subsection{Retro-projection filters}

Slide 20

\subsection{3D reconstruction}

Slide 21

We do the same thing for multiple planes, then calculate the inverse

Slide 22

\section{Wigner distribution, WD}

how can we use this same method for signal analysis, using the tomography as a tool that allows us to write the signal

\begin{align*}
W(x,y) &= \int_{\R} f (x + x'/2) f^{*}(x - x'/2) e^{-i2\pi u x'} \dd{x'}\\
&= \int_{\R} F (u + u'/2) F^{*}(u - u'/2) e^{-i2\pi u x'} \dd{u'}\\
\end{align*}

WD of stochastic processes.

\begin{equation}
W(\vec{r}, \vec{u}) = \iint_{\R^{2}}< f (\vec{r} + \vec{r'}/2) f^{*}(\vec{r} - \vec{r'}/2)>e^{-i2\pi \vec{u} \vec{r'}} \dd{\vec{r'}}
\end{equation}

\subsection{WD, coherencia}
analysis of non-stationary waves

The WD is a unified formalism of analysis and synthesis of signals both deterministic and stochastic.

\subsection{Rotation of WD during a Fourier transform}

See slide 24

\subsection{Radon-Wigner transform}

slide 25

Allow us to identify this type of signal, and for us to filter something something

Phase tomography. slide 26

\section{Conclusion}

Non-destructive reconstruction.

Iterative methods for reconstruction

Algorithms rely on whatever


\section{Homework resolution}

\subsection{Exercise sheet 2.15}

slide 28

\begin{equation}
f(x,y) = e^{-(x-x_{0})^{2} - y^{2}}
\end{equation}


Firstly, calculate the Radon transform of
\begin{equation}
f(x,y) = e^{-(x-x)^{2} - y^{2}}
\end{equation}

Then, apply a transformation






\end{document}
