% !TEX encoding = UTF-8 Unicode
% !TEX TS-program = pdflatexshell

\documentclass[../main/main.tex]{subfiles}

\begin{document}

Slides PRocesos aleatorios
\chapter{Random signals}

\section{Introduction}

Deterministic
\begin{itemize}
	\item The signal has the same value at all time
	\item Value at each instance can be defined
	\item Defined by a deterministic function $f(t)$
\end{itemize}

Stochastic
\begin{itemize}
	\item Signal follows a probability distribution
	\item The same instant could give a different measure
\end{itemize}


Our goal is to split a noisy signal into noise and signal.

\section{Continuous probability distributions}

\begin{equation}
	F_y (\nu) = P(y(t) \leq \nu),\, -\infty< \nu < \infty
\end{equation}

Probability is defined by the Probablity Density Function (PDF)  $f_y$

\begin{equation}
	f_y(\nu) = \frac{d F_y \nu}{\dd v}
\end{equation}

\section{Discrete probability }

Instead of PDF, we have a probability mass function.

\begin{equation}
	f_y(\nu) = P(y(t) = \nu),\,\nu \in \Z
\end{equation}

\section{Types of PDFs}


\subsection{Uniform}

\begin{equation}
	f_y(\nu)=
	\begin{cases}
		\frac{1}{b-a}, \textrm{for $x \in [a,b]$} \\
		0, \textrm{for $x \notin [a,b]$}          \\
	\end{cases}
\end{equation}


Our signal will be $x_i + \in_i$

\subsection{Normal or gaussian distribution}

\begin{equation}
	f_y(\nu)= \frac{e^ {\flatfrac{(\nu - \mu)^2}{2\sigma^2}}}{\sigma \sqrt{2\pi}}
\end{equation}



\subsection{Log-normal PDF}
A variable $x$ is log-normal distributed, if their logarithm, $\log x$ is normally distributed.

Common in microbiology and modeling of infection systems.
\begin{equation}
	f_y(\nu)= \frac{e^ {\flatfrac{(\log\nu - \mu)^2}{2\sigma^2}}}{\sigma \sqrt{2\pi}}
\end{equation}

\section{Characterization of system}

Slide 17

\subsection{Sample characterization}

Looking at the histogram, and obtaining statistics of values. You can use to estimate the different properties

\subsection{Average, expected value}

\begin{equation}
	\mu = \int_\R \nu f(\nu)\dd \nu
\end{equation}
$E(y)$ is an estimation of $\mu$. But in this class we use them interchangeably.

\subsection{Momentums}

\begin{equation}
	\mu_k'  = \int_\R \nu^k f(\nu)\dd \nu
\end{equation}

Central momentums

\begin{equation}
	\mu_k = \int_\R (\nu - \mu)^k f(\nu)\dd \nu
\end{equation}

\subsection{Mode, median, and mean}

The median and mean can be estimated more easily through the actual data.

However, the mode is better to calculate by first creating the theoretical probability distribution model and obtaining the $sup_\nu \{f(\nu)\}$

\subsection{Auto-correlation}
\subsubsection{Random signals}
Slide 23

Given some samples $X(i)$, we have $p(X(0), \ldots, X(10))$, for example.

\subsubsection{Definition}

Autocorrelation is the correlation of the signal with itself at a delay, $t_1 = t_2+\textrm{delay}$

\begin{equation}
	R(t_1, t_2) = E(y(t_1)y(t_2))
\end{equation}

We do this for each delay value.
\begin{equation}
	R(t_2+\textrm{delay}, t_2) = E(y(t_2+\textrm{delay})y(t_2))
\end{equation}
The value for $\textrm{delay}=0$ gives us the mean

Therefore, the autocorrelation is ploted in relation to the delay $R(\textrm{delay})$


\begin{itemize}
	\item The autocorrelation of a completely random variable would be zero
	\item The autocorrelation of a periodic signal would have peaks on the auto-correlation function at the delay values that are an integer multiple of the period of the signal
\end{itemize}

\section{Auto-covariance}

Similar

\begin{equation}
	C(t_1, t_2) = E((y(t_1)-\mu)(y(t_2)-\mu))
\end{equation}

We do this for each delay value.
\begin{equation}
	C(t_2+\textrm{delay}, t_2) = E((y(t_2+\textrm{delay})-\mu)(y(t_2)-\mu))
\end{equation}

Like before, The value for $\textrm{delay}=0$ gives us the variance

\subsection{Strict-sense stationary}

All statistical properties are invariant to a shift of the time origin.

See slide 27, that I didn't understand much

\subsection{White noise}
Slide 28

Is a signal whose autocorrelation is

\begin{equation}
	R(t_1,t_2) = I(t_1)\delta(t_1-t_2)
\end{equation}

\end{document}
