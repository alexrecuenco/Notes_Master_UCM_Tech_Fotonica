% !TEX encoding = UTF-8 Unicode
% !TEX TS-program = pdflatexshell

\documentclass[../main/main.tex]{subfiles}

\begin{document}
\chapter{Fabrication methods}

slide 5

Cleanroom.

\begin{itemize}
	\item Laminar air flow
	\item HEPA filter for the room
	\item Positive pressure room (prevent things from leaving)
	\item Entry through a 2 door system
\end{itemize}

Temperature control, particle\ldots In Europe, it is usual to aim for $20\pm 1 C$, humidity of $45\%\pm5\%$

slide 6

\begin{itemize}
	\item Large cost
	\item Maintainance
	\item Difficult daily use for workers
\end{itemize}

Habitual equipment. Note that you protect within it, since vapor from chemical or whatever could affect other parts.

slide 8

Diamond end with rotational system that print the edge.

You record it on metal to make the correct shape.

Less than $0.1 \mu m$ precission.

Rotational symmetry is quite decent with these, since rotating around an axis produces a very good trace

You can't mass-produce them like this, you create a negative master that is later used to create the different ones



slide 9

\section{Laser printing}

This needs to be on an isolated anti-vibration place. It has both passive and active anti-vibration methodology

For antivibration: Different anti-vibration fundation, on a passive vibration footing, with active vibration detectors that compensate

We use high-frequency lasers,  large lenses for focal,

Because we have a small focal point, but we have a ``profunddidad de foco'' that is very low. A very small margin of error. (2\mu m)

To prevent errors we use an autofocus, using a different laser that does not activate the resin to use to detect a margin of error and use as feedback to correct the height ($z$) of the laser

This type of fabrication needs to be really quick, since you need to do millions of perforations. For that reason the control system needs to go at a realy high frequency


\subsection{Resine}

slide 10

Photo-resine. A quemical substance that is sensitive and is placed over a surface. Then this is exposed to the laser, that with a very small power reacts

\begin{itemize}
	\item Blue light 365--405 nm.
	\item In the lab you need to use a yellow/red light to prevent affecting the resine
\end{itemize}

\subsection{Printing method}
slide 11

Positive (creates holes), or negative (add material)

Note that the width of the resine is comparable to the focal depth, and you therefore need to deal a lot with the light as it goes there, what is the interference inside that, etc.

\subsection{Photo-resin deposition}

slide 12

Leave a drop of resine on a spinning disk, spinning rate is important to spread it correctly

slide 13

The spinning can create different shapes, you want to find the right speed for it to prevent wave formation

slide 14 doesnt exist; slide 15

\subsection{Auto focus system}

slide 16

Pictures there of a red laser and and the blue laser. You get a signal method.

\subsection{Exposure}
slide 17

You need to calculate what is the exposure time you need based on laser power, etc.

slide 18

Gaussian beam exposition:

Convert the W in J by  integrating and decide that way what is the right width of these thigns

Width of the line is determined by the equation there with the log.

If we print the line with a gaussian beam we might create errors and ondulations at the edge of the line

Therefore, to create a better edge, it is better to create laser wave that has a sharp edge

Slide 19

Image of a line, started and accelerated.

To prevent the acceleration issues, you want to make it move at a constant speed by accelerating to the speed outside the area.

(TODO: Could we just analize acceleration and jerk to create a system that does not require step that accelerates)


Slide 20

This is an example, removing from the DOE


\section{Interferometry printing}

Slide 21

By using interferomtry we can print multiple much sharper lines at the same time with a period $p = \lambda/ 2 \sin (\theta)$

Slide 22

Creating a net with plane wave front interferomtry using a mirror.

You need to allow this to collimate from far away.

\section{Ablation laser}

More agressive. You ablate the material directly, so you need lasers with a lot more power.

These lasers are created as laser pulses.

Material in nanoseconds gets heated, melts, and explodes.


slide 24

thermical process

It is no longer a mW power laser, it is a 10W power or similar

Q-switching. Pulse frequency around 20 kHz.

This is a very direct technique but it works directly and much easier.

They are good for prototyping, and are flexible in use. They work both for metal and glass.

Slide 25

Lab here,

Describes teh current method. Using a non-linear crystal to split in 3, dissipating energy.

Slide 26

Description of lines over steel and glass.

If we have a 3$\mu$m focal point, but we want to make a 20$\mu m$ line. We can move the object further up, although the energy focus will be much smaller, so we will need to go slower.

In glass we can record inside of it by focusing inside. Look that the dots are created with a certain speed.

Slide 27

Fresnel lense over steel.

In a metal, the topography, even if polished, will have some ondulations.

You would use them for components that need to sustain large stresses and forces



\end{document}
