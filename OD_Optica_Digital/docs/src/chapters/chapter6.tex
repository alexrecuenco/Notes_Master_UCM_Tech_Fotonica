% !TEX encoding = UTF-8 Unicode
% !TEX TS-program = pdflatexshell

\documentclass[../main/main.tex]{subfiles}

\begin{document}


\chapter{Liquid crystals}
Slides `Cristales liquidos'

\section{Light polarization}

Light is actually a vector field.

\begin{align*}
\vec{E} &= \vec{E}_{0} e ^{i \vec{k}\cdot \vec{r} - wt }\\
\vec{B} &= \vec{B}_{0} e ^{i \vec{k}\cdot \vec{r} - wt }
\end{align*}

\begin{align*}
\vec{k}\cdot \vec{E_{0}} &= 0 \\
\vec{k}\cdot \vec{B_{0}} &= 0 \\
\vec{B_{0}} &= \frac 1 \omega \vec{k} \cross \vec{E_{0}} \\
\vec{E_{0}} &= \frac 1 \omega \vec{k} \cross \vec{B_{0}}
\end{align*}

We can assume without loss of generality $\vec{k} = (0,0,1)$, so that the only relevant components is $x$, and $y$. $E_{x}, E_{y}$. In that way we can check that, assuming that

\begin{align*}
E_{x} &= \mqty( \abs{E_{0x}} e ^{i \delta_{x}} \\ \abs{E_{0y}} e ^{i \delta_{y}} ) \\
 &  =e^{i \delta_{x}} \mqty( \abs{E_{0x}}  \\ \abs{E_{0y}} e ^{i ( \delta_{x} - \delta_{y})} ) \\
\end{align*}



\subsection{Anisotropic media}

Media in which the reaction to the .

For example, the amount of light reflected depends a lot on polarization, since the fresnel equation depends on the direction.

To define the polarization we have a 3-vector, however, we can use a O(3) rotation and place the $\vec{k}$ vector in the z axis, which creates a $(E_x, E_y)$ vector.

Each of these look like a $E_i(z,t) = E_{0,\,i}\cos{k z - \omega t + \delta_i}$

Since $k z - \omega t$ is shared between all directions, it does not contribute to polatization and we can study this looking just at the amplitude and dephase

Linear polarization, when the dephase is either 0 or $\pi$. The azimut on that case would be $\tan \alpha = \pm \frac{E_{0y}}{E_{0x}}$ which determines the angle it goes

Circular polarization. The dephase is $\pm \pi/2$ between both. $\delta_y - \delta_x = (2m + 1)\pi/2$


The jones vector determines the polarization uniquely in a complex unit vector.

(See equation slide 7)

\section{Polarizers}

There are certain materials that can change the polarization of an incoming beam.

We can describe any linear polarizer as a matrix $M$

Linear polarizer,


\begin{align*}
	\textrm{PL}(\theta = 0)       & = \ket{x}\bra{x} = \\
	\textrm{PL}(\theta = \pi / 2) & = \ket{y}\bra{y} =
\end{align*}

However a real one would have a minimum value, since it will be impossible to get zero

(See equation(s) slide 8)


\subsection{Rotating polarizers}

\begin{equation}
	P(\theta) = R(-\theta) P(0) R(\theta)
\end{equation}

Think of it as ``rotate and apply then unrotate''

Note that for a 45 degree angle the matrix is a 1/2,1/2,1/2,1/2

\subsection{\emph{Retardadores}}

Apply a phase difference between both parts.

\section{Applying multiple polarizers}

Slide 10

When light goes through multiple polarizers, the polarizing state at the end can be understood as one polarizer

\begin{equation}
	\ket{s}  = L_N \cdots L_2 L_1 \ket{e}
\end{equation}

Note that the first polarizer in the system goes on the right, and also note that polarizers do not commute in general.

Therefore, we can define it $L_{\textrm{total}} = L_N \cdots L_2 L_1$


\section{Anysotropy}

slide 11

An anysotropy matrix is one that behaves differently for different components

For $x$ we use the $n_o$ for $y$ we use $n_e$. The material changes the phase differently for each component.

$\beta = \phi_y - \phi_x = \frac{2 \pi d (n_e - n_0)}{\lambda_0}$ is the dephase applied by the material.


\section{Liquid crystal}

\subsection{Types}

slide 12

\subsubsection{Isotropic}

even if each molecule is anysotropic, when everything is perfectly disordered, the media becomes isotropic, since each applies a random phase. A good example is water (polar molecule that is disordered)

A liquid crystal aligns itself when you apply an electric field to it. Therefore we can obtain

Esmectic and nematic depending on whether the voltate is large or low

\subsection{TFT, Thin Film Transistor}
slide 13

Some explanation of hte material I didn't understand.

You need materials that are both transparent and conductors. That way light can go through it, and you can apply a voltage across to get a polarization

\subsection{LCOS, Liquid crystal on silicon}

slide 14

You have a cover, a liquid crystal, and all the electronics are behind the liquid crystal, creating a reflection.

the main advantage with respect to TFT, the light goes twice through the material, so we get twice the 'whatever we want' with the same width

A very important advantage is that the opaque part is not obstructing our view, so we get a 100\% of light getting used, while on the other case a part of the light is absorved on the electronics.

\section{Matrix model}
slide 15

Rigurous model with Mueller matrices that are 4-by-4

The study of these liquid cristals is important, we will summarize the papers that are described here for the 90s


\subsection{Saleh-Lu mode}

slide 16

Liquid cristal in-between two glass.

Molecules align to the walls

`birrefringencia', see equation at the top.

We will divide these in $N$ layers that each have this `birrefringencia' (we will later make this $N$ tend to infinity)

Each layer will have an angle that is linear from 0 to the final twist angle of the `placas'. $\tau_i =  i \alpha / N$

This angle in eq top slide 16 is the same as the $\delta$

We can describe these with a matrix of the actual media $S(angle I dont know how to type)$. Each of these create a rotation

We therefore can describe the total media with a matrix multiplication

Then, all these angles, since they are linearly different $R(\tau_i)R^-1(\tau_{i-1}) = R(\Delta \tau) = R(\alpha / N)$

\begin{equation}
	M = R^{-1} \left[S R(\alpha/N)\right]^N
\end{equation}

slide 18

How do we solve $A^N$? solution at the end using Cayley-Hamilton theorem

Then we take this to infinity

slide 19

Solution of this

At the end $\alpha$ is the total twist we obtain. Note that if the $\alpha = 0$ we recover the starting equation, since there wouldn't be any change.


We initially used $\Psi = 0$, which is the angle of the entry layer. The entry layer is not necessarily aligned with the polarized entry light. So we also include the $\Psi$ rotation

This is what happens without a voltage

slide 20

What happens when you apply a voltage? With a large enough voltage, the molecules are directed on a certain direction, with enough voltage it will align with the field and not disrupt it.

Slide 21, you can see how when you apply a voltage the molecules start aligning and "lifting". The $\xi$ angle changes, and the effective index of refraction changes.

The `birrefringencia' is proportional to the index of refraction,

slide 23

When we don't have a voltage

Models

Berreman physical model

Saleh-Lu

Simplify to a linear model for twist angle,

Create a linear model between high and low model across the entire

Coy model

Create a linear but with a zero threshold.

Create a depth of diffrengigasg of width $m$ on 2 parts, so the matrices of those stages are different

Marquez model, a better model.

Slide 24 TODO: READ on your own time

Slide 25--26 models with pretty euqations.

Slide 27

When you calculate it experimentally, you can calculate the $\beta$ and $\delta$, these depend on voltage and that determines your system.

Slide 28

Fit these models to the theoretical model

Slide 29

you have a liquid crystal and two polarizers. In the middle you have electrodes. You will be able to graduate with voltage in ``amplitude'' mode.

When it passes by a $\lambda/4$ polarizer it becomes circular. It will dephase the light. We will modulate the change of phase of the light that is outgoing.

We will obtain therefore masks of amplitude or masks of phase

\end{document}
