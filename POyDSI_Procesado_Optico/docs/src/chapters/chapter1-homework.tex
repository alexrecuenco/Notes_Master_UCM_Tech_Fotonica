% !TEX encoding = UTF-8 Unicode
% !TEX TS-program = pdflatexshell

\documentclass[../main/main.tex]{subfiles}

\begin{document}
\section{Homework}

20 de septiembre: entrega de problemas 1.2(e,f), 1.3 y 1.6 (un único documento pdf)
\subsection*{Problem 1.2.e.}

\begin{align*}
\int _{-\infty}^{\infty}f(x) \delta (x^{2} - a^{2}) d x  &= \\
\textrm{(\small since $x^{2'} = 2x$ and $x=\pm a$ are the poles)}
	&= \int _{-\infty}^{\infty}f(x)\left( \frac{\delta (x-a)}{|2x|} + \frac{\delta(x+a)}{|2x|}\right) d x\\
\textrm{(\small evaluating the distribution)}
	&=\frac{f(a)}{|2a|} + \frac{f(-a)}{|2a|}\\
\textrm{(\small simplifying)}
	&= \frac{f(a) + f(-a)}{|2a|}\\ \qed
\end{align*}

\subsection*{Problem 1.2.f.}

\begin{align*}
\int _{-\infty}^{\infty}f(x) \left[\delta(x-a) - \delta(x+a)\right]d x  &= \\
\textrm{(\small evaluating and simplifying)}
	&=f(a) - f(-a) \\
	\qed
\end{align*}

\pagebreak
\subsection*{Problem 1.3}

Draw a squema of 

\begin{align*}
e^{-x^{2}/4}\left(\comb_{1}(x) + comb_{1.5}(x)\right)&\\
&=e^{-x^{2}/4}\sum_{n=-\infty}^{\infty}\left(\delta(x-n) +\delta(x-1.5n)\right)
\end{align*}

Therefore, this is a discretized function, where the pole size is proportional to the gaussian  $e^{-x^{2}/4}$ on all integers and multiples of $1.5n$ except in those integers in which both conditions are met (the multiples of 3), the pole size is twice that of the gaussian instead. 

In the following drawing those that get a double contribution are marked in red and the heights of the arrows represent the pole weight.

\begin{center}
	\pgfmathsetmacro{\h}{1}
	\pgfmathsetmacro{\k}{0.3}
	\pgfmathsetmacro{\T}{1}
	\pgfmathsetmacro{\Tminusk}{\T-\k}
    \begin{tikzpicture}[x=1cm, y=2cm]
    % grid
	\draw[help lines, dotted] (-4,-1) grid (4, 3);
        \draw[thin](-4,0)--(4,0);
        \draw[thin](0,-1)--(0,3);
        \foreach \x [evaluate={\hval=exp(-(3*\x-1)*(3*\x-1)/4)}] in {-1, ..., 1} {
		\draw[->, very thick] (3*\x-1, 0) --+(0, \hval) ;
        };              
         \foreach \x [evaluate={\hval=exp(-(3*\x+1)*(3*\x+1)/4)}] in {-1, ..., 1} {
		\draw[->, very thick] (3*\x+1, 0) --+(0, \hval) ;
        };       
        \foreach \x [evaluate={\hval=exp(-\x*\x/4*9)}] in {-1, ..., 1} {
		\draw[->, very thick, red] (3*\x, 0) --+(0, 2*\hval) ;
        };        
        \foreach \x [evaluate={\hval=exp(-(3*\x+1.5)*(3*\x+1.5)/4)}] in {-2, ..., 1} {
		\draw[->, very thick] (3*\x+1.5, 0) --+(0, \hval) ;
        };
        
        	\draw [|<->|] (0,-0.05) 
		--++(1,0) node[pos=0.5, anchor = north] {+1n};
        	\draw [|<->|] ++(0,-0.3) 
		--++(1.5,0) node[pos=0.5, anchor = north] {+1.5n};
   	
	\draw [|<->|] ++(0,-0.6) 
		--++(+3,0) node[pos=0.5, anchor = north] {+3n};

        
    \end{tikzpicture}
\end{center}

\subsection*{Problem 1.6}

Express the following signal as a Fourier series in 3 different ways.
\begin{center}
	\pgfmathsetmacro{\h}{1}
	\pgfmathsetmacro{\k}{0.5}
	\pgfmathsetmacro{\T}{1}
	\pgfmathsetmacro{\Tminusk}{\T-\k}
    \begin{tikzpicture}
    % grid
	\draw[help lines, dotted] (-3,-1) grid (3, 2);
        \draw(-3,0)--(3,0);
        \draw(0,-1)--(0,2);
     % drawing
     % This can be more clean with \foreach \x in {0, ... , 5} homework for the reader
     	\draw[very thick] (-3+\k/2, 0) --
                		++(\Tminusk, 0)  node(abelow){} --
                		++(0, \h) node (a){} --
                		++(\k, 0) node(b){} --
                		++(0, -\h) --
                		++(\Tminusk, 0)  node(bbelow){}--
                		++(0, \h) --
                		++(\k, 0) --
                		++(0, -\h) --
                		++(\Tminusk, 0) --
                		++(0, \h) --
                		++(\k, 0) --
                		++(0, -\h) --
                		++(\Tminusk, 0) --
                		++(0, \h) --
                		++(\k, 0) --
                		++(0, -\h) --
		         ++(\Tminusk, 0) --
                		++(0, \h) --
                		++(\k, 0) node(c){} --
                		++(0, -\h) node(d){} --
		(3, 0);
	\draw [|<->|] (a.north) --(b.north) node[pos=0.5, anchor = south] {k};
	\draw [|<->|] (abelow.south) --(bbelow.south) node[pos=0.5, anchor = north] {T};
	\draw [|<->|] (c.east) --(d.east) node[pos=0.5, anchor = west] {h};
		
    \end{tikzpicture}
\end{center}


\begin{enumerate}
	\item $F(x) = $
	
\end{enumerate}•
\end{document}
