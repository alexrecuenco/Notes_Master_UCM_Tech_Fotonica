% !TEX encoding = UTF-8 Unicode
% !TEX TS-program = pdflatexshell

\documentclass[../main/main.tex]{subfiles}

\begin{document}

\chapter{Holography and numerical calculus of Fourier Transform and diffracion integrals}
Slide 3

\begin{itemize}
	\item Obtain amplitude and phase information
	\item Two steps
	      \begin{enumerate}
		      \item Formation and registration of interference fringes (hologram)
		      \item Reconstruction of object beam from hologram
	      \end{enumerate}

\end{itemize}


\section{Different configurations}

Slide 7

In-line holography. If the reference wave is parallel to the object

Off-line holography. Reference wave is inclined with respect to the object one. Highly coherent ilumination is required.


Slide 8


Fourier domain filtering (With a 4f system, you can filter in the middle the fourier transform of the system)

We filter what we don't want. We can then incline and get the fringes
Slide 9

Gabor holography

Object and reference are on the same plane, you can't do the same as with the previous one


Slide 10

With an opaque F in a white background,

Can I do the same reconstruction if it is a transparent F on an opaque screen?
Answer is no, reference is on the same line shown and reconstruction is harder

Slide 11

How to fix this issue?

You make the reference wave separate to the holographic wave

In that case, it can reconstruct better both images

\section{Phase recovery interpherometry }
Slide 12

Michelson

Superposition of object wave with reference wave [eq 1]

We can recover phase, controlling phase shifting.
We can recover phi measuring 4 images at 4 different phase shifts. (See [eq 2] and [eq 3])

We need to be able to move the object with a lot of precision

\section{Digital Holography advantages}

slide 14

\begin{itemize}
	\item Reconstruction of object phase, quantitative phase images
	\item Object refractive index recovery. RI instead of staining
	\item Reconstructing with reading beams.
	\item
\end{itemize}


slide 15

Digital creation of interference.

Circular beams with same shape and different phases.

These would provide different things to trapped particles.
Smooth phase change would give continuous, while phase change jerkiness would cause acceleration/deceleration

Generate the beam digitally and reproduce it analogously

Slide 16. Experiemnt on trapping using these.


TFM. Digital holigraphy, thinguies and other things. Could be a Master thesis if talked about.


\section{Numerical methods for Fourier transform}

sllide 20

\subsection{FFT}


Riemann integral,

Discrete Fourier transform

Inverse.

DFT transform is a matrix, $O(N^2)$

FFT is $N \log_2(N)$


How?

Divide FFT into odd/even, then calculate the FFT of each separately.

Repeat this step recursively (slides up to 23)

fft shift?


Slide 28

Diffraction integral simulation

Fresnel transform, Impulse response method, angular spectrum method (Transfer function method)

Direct method of the integral calculation

Fresnel Transform MEthod (FTM)
Slide 29 --33

Single FT involved

Object in an rectangular area, $L \times  L$ $M \times N$ pixels. Assume $M=N$

Wavelength $\lambda$

One pixel correspnds to $\deltax = \lambda z / L$

Field of view depends on $z$

Aoid aliasing by $z_{\min} = L^2/ N \lambda$

Slide 32, calculation for the letter F

Problem: Window size grows linearly with distance


\section{Impulse Response method}

Based on FFT

Three FT are required.

Pixel resolution does not depend on propagation distance, since it is based on inverse FT

Slide 36, showing how it works for smaller z


\section{Transfer function}

Slide 37

Fresnel transform transfer function calculated analitically

Transfer function in paraxial approximation.

Two FT are required

Size of window is constant

Slide 38 shows how it is not limited by a minimum $z$ value

Slide 39--40 code in matlab


Slide 41

What method to use.

Short distance, TF and IR. TF limitation $D \leq \lambda z / Delta x$

Further passed this limit TF attenuates the field.

The IR creates copies of the observtion plane field,


For critical samplling, TF requires only two FFT

For long distance



\subsection{Phase unwrapping}

Important for topographic measures

How is the phase calculated? In the range $[-\pi, \pi]$. So there will be jumps, we neeed to unwrap those jumps.

Slide 43

Phase unwrapping:

Start from the left, and whenver there is a jump you look at the sign of the change and change everything moving forward

Slide 44 (shown in picture)

Slide 45 Phase unwrapping with white noise

It can create wrong jumps in our signal

Slide 46 Phase unwrapping of undersampled signal.


\subsection{Optical phase unwrapping OPU}
Slide 47

2D signals unwrapping is computationally demanding.

OPU provides better


Proffesor recommends from references

D. Voelz, Computational Fourier Optics, v. TT89, SPIE tutorials, Washington, USA (2011).



Problem 1.10 $f' = f \sin(2 \pi \u_0 x)$

We can use this for super-resolution. If we have an object iluminated by a sinusoidal signal,

The convolution is equivalent with two displacements, which gives us extra resolution on frequencies, almost double.



Problem 2.8

Which filters let us observe the edges of an image?

First, higher frequencies provide the information of the edges of signals. Therefore,

\begin{enumerate}
	\item No, for normal images $\abs{TF}$ is bigger for lower frequencies

	Therefore, we will be enhancing the low  ones and removing the higher ones.


	\item Yes, by removing the modulus and leaving just the phase, it will provide more visibility to higher frequencies
	\item Yes, by dividing it should provide us with better image quality
	\item No, it is a blurring effect, it is a high pa
	\item No, it is a blur as well
\end{enumerate}


Homework 1.6
\end{document}
